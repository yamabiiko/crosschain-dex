% TeX root = main.tex

\section{Atomic Swaps and Existing Solutions}\label{sec:atomic_swap_overview}

\subsection{Existing solutions}

\subsubsection{Hash Time Lock Contracts}

A Hash Time-Lock Contract (HTLC) is a contract that enables conditional payment based on the revelation of a cryptographic secret within a certain time window. Formally, an HTLC is characterized by a tuple  $(\mathsf{amnt_a}, h, T, \mathsf{pk_0}, \mathsf{pk_1})$ where
\begin{itemize}
	\item $\mathsf{amnt_a}$ denotes the amount of $\mathsf{a}$ assets to be exchanged
	\item $h$ is the hash of a secret value $r$, i.e., $h = \mathcal{H}(r)$ for some cryptographic hash function $\mathcal{H}$
	\item $T$ is a timelock parameter, typically a block height or timestamp, indicating the deadline after which funds can be refunded
	\item $\mathsf{pk_0}$ is the public key of the sender (who can reclaim the funds after the timeout).
	\item $\mathsf{pk_1}$ is the public key of the intended recipient (who can claim the funds upon presenting the correct preimage $r$ before the timeout).
\end{itemize}

The HTLC transfers $\mathsf{amnt_a}$ to $\mathsf{pk_1}$ if invoked before timeout $T$ with input value $r$ such that $\mathcal{H}(r) = h$. 
If the contract is invoked after timeout $T$, it refunds the assets $\mathsf{amnt_a}$ to $\mathsf{pk_0}$ unconditionally.

Using HTLCs as a building block, an atomic swap protocol can be constructed as follows: \\
1) Alice chooses $r$, computes $h = \mathcal{H}(r)$, transfers $\mathsf{amnt_a}$ into an $(\mathsf{amnt_a}, h, T_0, \mathsf{pk_0}, \mathsf{pk_1})$ on blockchain $\bca$ and sends $h,T$ to Bob. \\
2) Bob finishes the setup of the exchange by choosing a time $T_1 < T_0$ and transferring his $\mathsf{amnt_b}$ assets into an HTLC$(\mathsf{amnt_b}, h, T_1, \mathsf{pk_{tx}}, \mathsf{pk_{rx}})$ on blockchain $\bcb$.

HTLCs serve as the core building block in many atomic swap protocols. A standard two-party protocol between Alice and Bob proceeds as follows:

Alice chooses a uniformly random secret value $r$, computes $h = \mathcal{H}(r)$, and initiates an HTLC on blockchain $\bca$ locking $\mathsf{amnt_a}$ to Bob under the tuple $(\mathsf{amnt_a}, h, T_0, \mathsf{pk}\text{Alice}, \mathsf{pk}\text{Bob})$. She then sends $(h, T_0)$ to Bob.

Bob selects a smaller timeout $T_1 < T_0$ and sets up an HTLC on blockchain $\bcb$ locking his $\mathsf{amnt_b}$ to Alice under $(\mathsf{amnt_b}, h, T_1, \mathsf{pk}\text{Bob}, \mathsf{pk}\text{Alice})$.

Alice redeems the funds from Bob’s HTLC on $\bcb$ by revealing $r$. Since transactions on public blockchains are visible, Bob can then observe $r$ on-chain and use it to redeem the funds from Alice’s HTLC on $\bca$ before her timeout $T_0$.

This protocol ensures atomicity: either both parties receive their respective assets, or after the timeout, each party reclaims their original funds.

While HTLCs are widely used in practice, especially in early cross-chain systems and off-chain protocols such as the Lightning Network, they suffer from several critical limitations, especially when applied in a privacy-preserving or cross-paradigm setting.

\paragraph*{Compatibility of hash functions}
HTLC-based atomic swaps require both blockchains involved to support the same hash function. For instance, if chain $\bca$ supports SHA-256 and chain $\bcb$ supports Blake2, an HTLC constructed with a hash $h = \mathcal{H}(r)$ cannot be replicated on the second chain unless both parties can compute and verify the same preimage relation. One possible workaround is to have Alice compute $h = \mathcal{H}(r)$ and $h' = \mathcal{H}'(r)$ and publish a non-interactive zero-knowledge (NIZK) proof that both hashes are computed from the same preimage. However, this may increase protocol complexity and setup cost, and depends on the availability of NIZK-friendly primitives on both chains. Since the same preimage $r$ is reused on both chains, an external observer can trivially link the two transactions, breaking the unlinkability that privacy-preserving systems aim to provide. A workaround—such as committing to a shifted value $r + r'$ with an accompanying NIZK proof—adds cryptographic overhead and requires careful implementation to avoid privacy leaks.
\paragraph*{(In)compatibility with private ledgers}
Many privacy-preserving cryptocurrencies, such as Monero or Zcash, do not support expressive scripting or global hash preimage verifiability. Even when HTLCs are theoretically implementable (e.g., in a limited form over RingCT), their transaction structure becomes easily distinguishable from standard private transfers, harming plausible deniability. In such cases, the protocol becomes asymmetric: the public-chain participant must move first, revealing the preimage, which the private-chain user can then observe off-chain—this violates the symmetry typically desired in fair exchange protocols.
\paragraph*{Miner incentives attacks} HTLCs may also suffer from incentive misalignments, particularly in adversarial mining environments. Miners observing the hash preimage during the redemption phase may attempt to front-run or extract value, especially when the reward from a successful claim is higher than the block reward or other fees. These risks have been documented in Kolluri et al., 2022, which explores protocol designs vulnerable to such "griefing" attacks in HTLC settings.

\subsubsection{Universal Atomic Swaps}
\begin{itemize}
\item this needs to be a little rewritten, pointing out on the impracticaly of the protocol due to ASICs
\item also mention the "bug" of the proposed protocol due to not considering chain's confirmation time (might be used as motivation for our blockchain interface?)
\end{itemize}

Thyagarajan et al. \cite{uas} introduced one of the first atomic swap protocols substituting blockchain timelocks with a cryptographic primitive. 
The core building block utilized are Verifiable Timed Signatures (VTS) \cite{vts}, which lets a user generate a timed commitment $C$ of a signature $\sigma$ on a message $m$ under a public key $\mathsf{pk}$. The commitment $C $ must hide the signature $\sigma$ for time $\mathsf{T}$ and producing a proof $\pi$ that $C$ contains a valid signature $\sigma$. This ensures that $\sigma$ can be publicly recovered in time $\mathsf{T}$ by anyone who solves the computational puzzle. We note that a similar construction called Verifiable Timed Discretelog (VTD) allows to commit on a dlog value instead of a signature, and can be alternatively used in the protocol.

Let $P_0$ and $P_1$ be two parties where $P_0$ wants to exchange $\mathsf{amnt_a}$ on blockchain $\mathbb{A}$ from their address $\mathsf{pk_{init(\mathbb{A})}}$ for $\mathsf{amnt_b}$ on blockchain $\mathbb{B}$ to $\mathsf{pk_{swp(\mathbb{B})}}$ and vice-versa for $P_1$.

In the setup phase of the protocol, the parties run a 2PC protocol to setup two freeze addresses on the respective chains $\mathsf{pk_{frz(\mathbb{A})}}$ and $\mathsf{pk_{frz(\mathbb{B})}}$, where each party posseses one share of the respective secret keys, e.g. $\mathsf{sk_{frz(\mathbb{A})}} := \mathsf{sk_{frz0}} \oplus  \mathsf{sk_{frz1}}$. \\
Now the parties create a refund transaction transferring back the assets in case of timeout, for $P_0$ we have $\mathsf{tx_{rfnd(\mathbb{A})}}$ refunding $\mathsf{amnt_a}$ from $\mathsf{pk_{frz(\mathbb{A})}}$ to $\mathsf{pk_{init(\mathbb{A})}}$ and similarly for $P_1$ $\mathsf{tx_{rfnd(\mathbb{B})}}$. \\
Each party generates a timed commitment on the signature of the counterparty's refund transaction, where $P_0$ receives a $\mathsf{VTS}$ with commitment $C_0$ and timeout $T_0 = T_1 + \Delta$ and $P_1$ receives a $\mathsf{VTS}$ with commitment $C_1$ and timeout $T_1$. Once both $\mathsf{VTS}$ commitment are verified the parties proceed to transfer the assets to the freeze addresses, assured to retrieve the refund transaction signatures after force opening the commitments with timeouts $T_0$ and $T_1$.

In the subsequential lock phase, parties first initialize the swap transactions $\mathsf{tx_{swp}}$ transferring $\mathsf{amnt}$ from $\mathsf{pk_{frz}}$ to $\mathsf{pk_{swp}}$ for the respective chains. They then compute via 2PC $\mathsf{lk} := \sigma_{\mathsf{swp}(\mathbb{A})} \oplus \mathcal{H}(\sigma_{\mathsf{swp}(\mathbb{B})})$, where  $P_0$ receives $\sigma_{\mathsf{swp}(\mathbb{B})}$ and $P_1$ receives $\mathsf{lk}$.  When $P_0$ publishes $\mathsf{tx_{swp(\mathbb{B})}}$ together with  $\sigma_{\mathsf{swp}(\mathbb{B})}$, $P_1$ can unmask $\mathsf{lk}$ by computing $\mathcal{H}(\sigma_{\mathsf{swp}(\mathbb{B})})$ to retrieve $\sigma_{\mathsf{swp}(\mathbb{A})}$ and publish $\mathsf{tx_{swp(\mathbb{A})}}$.

 If $P_0$ fails to publish $\mathsf{tx_{swp(\mathbb{B})}}$ before $T_1$, $P_1$ will publish the refund transaction $\mathsf{tx_{rfnd(\mathbb{B})}}$ and similarly for $P_0$ if $P_1$ timeouts during the protocol execution.

Note that parties must also take into account potential differences in the computational power available for force opening the VTS commitments. This prevent scenarios where one party force opens its VTS commitments earlier than expected, potentially stealing 
 the other party's assets during the swap lock or complete phase. Therefore,  $\Delta$ (such that T0 = T1 + $\Delta$) must be large enough to tolerate time differences in opening the VTS commitments. \\
