\documentclass{article}      	% Style of the document                     
\usepackage{fullpage}
\usepackage{amsmath,amsthm}     	   	% Maths    
\newtheorem{definition}{Definition}                                      
\usepackage[utf8]{inputenc}	% UTF-8 characters                                               
\usepackage[T1]{fontenc}    	% Tuki ääkkösille (Finnish names don't cause problems)                                            
\usepackage{parskip}        		% Linebreak between paragraphs                
\usepackage{svg}
\usepackage{graphicx}       		% Graphics package for adding figures                        
\usepackage{epstopdf}       		% Possibility to add *.eps figures
\usepackage{ dsfont }            % Symbol for real numbers
\usepackage{extarrows}
\usepackage{float}
\usepackage{makeidx}
\usepackage[ruled,vlined]{algorithm2e}
\usepackage[margin=1.7cm]{geometry}
\usepackage{amsmath,amssymb,mathtools}
\usepackage{microtype}
\usepackage{tcolorbox}
\tcbset{colback=white,colframe=black!70,boxrule=0.4pt,arc=1.2mm,left=3pt,right=3pt,top=2pt,bottom=2pt}
\usepackage{titlesec}
\usepackage{enumitem}
\usepackage{inconsolata}
\newcommand{\roundlabel}[1]{\texttt{\scshape #1}}
\setlist[itemize]{leftmargin=1.1em,itemsep=0.2ex,topsep=0.2ex}
\setlength{\parindent}{0pt}
\setlength{\parskip}{4pt}

% Small caps heading for each algorithm name
\newcommand{\procname}[1]{\textbf{\textsf{#1}}}
% Convenience
\newcommand{\N}{\mathbb{N}}

% A compact pseudo “algorithm” environment using tcolorbox
\newenvironment{algo}[1]{%
  \begin{tcolorbox}
  \textbf{#1}\par\vspace{2pt}\hrule\vspace{4pt}
}{%
  \end{tcolorbox}
}
\usepackage{enumitem}        % possibility to label list items by alphabet
\newcommand{\M}[1]{\ensuremath{\text{\texttt{#1}}}}
\usepackage[
    lambda,
    operators,
    advantage,
    sets,
    adversary,
    landau,
    probability,
    notions,
    logic,
    ff,
    mm,
    primitives,
    events,
    complexity,
    asymptotics,
    keys]{cryptocode}

% \usepackage{todonotes}
\usepackage{tcolorbox}

\usepackage{amsmath,amsfonts,graphicx,amssymb,amsthm}


\usepackage[bookmarksdepth=2,draft=false]{hyperref}
\hypersetup{colorlinks=true,linkcolor={red!50!black},citecolor=darkgray,linkcolor=darkgray}
\usepackage[capitalize]{cleveref}

\usepackage[draft]{comments-bugs-todos}

\newcommand{\rnote}[1]{{\color{purple} Reyhaneh: #1}}

% TeX root = main.tex

%% general
\mathchardef\mhyphen="2D
\newcommand{\fdv}{\mathcal{F}}
\newcommand{\tdv}{\mathcal{T}}
\newcommand{\vdv}{\mathcal{V}}
\newcommand{\cX}{\mathcal{X}}
\newcommand{\cF}{\mathcal{F}}
\newcommand{\cG}{\mathcal{G}}
\newcommand{\ID}{\mathcal{I}}
\newcommand{\bits}[1][]{\{0,1\}^{#1}}
\renewcommand{\vec}[1]{\mathbf{#1}}
\newcommand{\mat}[1]{\mathbf{#1}}
\newcommand{\inner}[2]{\langle #1, #2 \rangle}
\newcommand{\transpose}{\mathtt{T}}
\newcommand{\round}[1]{\lfloor #1 \rceil}
\renewcommand{\dist}{\mathsf{dist}}
\renewcommand{\Pr}[2][]{{\text{Pr}_{#1}\left[#2\right]}}
\newcommand{\Exp}[2][]{{\mathbb{E}_{#1}\left[#2\right]}}
\newcommand{\mathcm}[2][1cm]{\hspace{#1}{\mbox{/\!\!/ } \text{\scriptsize#2}}}

%% lattice problems
\newcommand{\SIS}{\mathsf{SIS}}
\newcommand{\ISIS}{\mathsf{ISIS}}
\newcommand{\nfSIS}{\mathsf{nfSIS}}
\newcommand{\dSIS}{\mathsf{dSIS}}
\newcommand{\LWE}{\mathsf{LWE}}
\newcommand{\nfLWE}{\mathsf{nfLWE}}
\newcommand{\nfdLWE}{\mathsf{nfdLWE}}
\newcommand{\sLWE}{\mathsf{sLWE}}
\newcommand{\dLWE}{\mathsf{dLWE}}
\newcommand{\SVP}{\mathsf{SVP}}
\newcommand{\CVP}{\mathsf{CVP}}
\newcommand{\SIVP}{\mathsf{SIVP}}
\newcommand{\GapSVP}{\mathsf{GapSVP}}
\newcommand{\BDD}{\mathsf{BDD}}
\newcommand{\NTRU}{\mathsf{NTRU}}
\newcommand{\sNTRU}{\mathsf{sNTRU}}
\newcommand{\dNTRU}{\mathsf{dNTRU}}

%% lattice macros
\newcommand{\TT}{\mathbb{T}}
\newcommand{\ring}{\mathcal{R}}
\newcommand{\lattice}{\mathcal{L}}
\newcommand{\piped}{\mathcal{P}}
\newcommand{\ball}{\mathcal{B}}
\newcommand{\Hyb}{\mathsf{Hyb}}
\newcommand{\lspan}{\mathsf{span}}
\newcommand{\rank}{\mathsf{rank}}
\newcommand{\lsb}{\mathsf{LSB}}
\newcommand{\pubparam}{\mathsf{pp}}

%% group macros

%% syntax
\newcommand{\mpk}{\mathsf{mpk}}
\newcommand{\msk}{\mathsf{msk}}
\newcommand{\msg}{\mathsf{msg}}
\newcommand{\rnd}{\mathsf{rnd}}
\newcommand{\ctxt}{\mathsf{ctxt}}
\newcommand{\com}{\mathsf{com}}
\newcommand{\td}{\mathsf{td}}
\newcommand{\id}{\mathsf{id}}
\newcommand{\stmt}{\mathsf{stmt}}
\newcommand{\wit}{\mathsf{wit}}
\newcommand{\tx}{\mathsf{tx}}
\newcommand{\aux}{\mathsf{aux}}
\newcommand{\ek}{\mathsf{ek}}

\newcommand{\Setup}{\mathsf{Setup}}
\newcommand{\Commit}{\mathsf{Com}}
\newcommand{\TrapGen}{\mathsf{TrapGen}}
\newcommand{\SampD}{\mathsf{SampD}}
\newcommand{\SampPre}{\mathsf{SampPre}}
\newcommand{\Prove}{\mathsf{Prove}}
\newcommand{\Verify}{\mathsf{Verify}}
\newcommand{\val}{\mathsf{val}}

%% primitive/scheme name
\newcommand{\PKE}{\mathsf{PKE}}
\newcommand{\LTDF}{\mathsf{LTDF}}
\newcommand{\rsagen}{\mathsf{RSAGen}}
\newcommand{\rsa}{\mathsf{RSA}}
\newcommand{\LHE}{\mathsf{LHE}}
\newcommand{\C}{\mathcal{CS}}
\newcommand{\NTRUEncrypt}{\mathsf{NTRUEncrypt}}

%% others
\newcommand{\oracle}{\mathcal{O}}
\newcommand{\pcas}{~\mathbf{as}~}

\newcommand{\polylog}[1][\secpar]{\mathsf{polylog}(#1)}

\newcommand{\indrsidcpa}{\mathrm{IND\$}\mhyphen\mathrm{sID}\mhyphen\mathrm{CPA}}
%\newcommand{\oplus}{\, \texttt{XOR} \,} % shorthand for typing the XOR operator in mathmode


% CommitTx
\newcommand{\amnt}{\mathsf{amnt}}
\newcommand{\ac}{\mathsf{ac}}
\newcommand{\accd}{\mathsf{accd}}
\newcommand{\ssk}{\mathsf{ssk}}
\newcommand{\tsk}{\mathsf{tsk}}
\newcommand{\rsk}{\mathsf{rsk}}
\newcommand{\tk}{\mathsf{tk}}
\newcommand{\spk}{\mathsf{spk}}
\newcommand{\tpk}{\mathsf{tpk}}
\newcommand{\rpk}{\mathsf{rpk}}
\renewcommand{\pk}{\mathsf{pk}}
\renewcommand{\sk}{\mathsf{sk}}
\renewcommand{\time}{\mathsf{time}}
\newcommand{\tout}{\mathsf{tout}}
\newcommand{\parse}{\mathbf{parse} \:}
\newcommand{\vft}{\mathsf{vft}}
\newcommand{\atype}{\mathsf{atype}}
\newcommand{\expired}{e}
\renewcommand{\pp}{\mathsf{pp}}
\newcommand{\bca}{\mathbb{A}}
\newcommand{\bcb}{\mathbb{B}}
\newcommand{\pki}{\pk_{\mathsf{i}}}
\newcommand{\ski}{\sk_{\mathsf{i}}}
\newcommand{\pkm}{\pk_{\mathsf{m}}}
\newcommand{\skm}{\sk_{\mathsf{m}}}
\newcommand{\pkc}{\pk_{\mathsf{c}}}
\newcommand{\skc}{\sk_{\mathsf{c}}}
\newcommand{\pkr}{\pk_{\mathsf{r}}}
\newcommand{\skr}{\sk_{\mathsf{r}}}
\newcommand{\pks}{\pk_{\mathsf{s}}}
\newcommand{\sks}{\sk_{\mathsf{s}}}
%\newcommand{\tx}{\mathsf{tx}}
%\newcommand{\st}{\mathsf{st}}
\errorcontextlines=10




\usepackage{tikz}
\usetikzlibrary{decorations.pathreplacing}
\usetikzlibrary{decorations.pathmorphing}


\definecolor{cgreen}{RGB}{0, 153, 51}
\definecolor{cblue}{RGB}{0, 102, 204}
\definecolor{cyellow}{RGB}{255, 204, 0} 
\definecolor{cred}{RGB}{204, 51, 0} 

\newtcolorbox{todobox}{colback=yellow!3!white, colframe=white!75!black}

\newcommand{\commentline}[2]{%
    \tikz[remember picture, overlay]{
        \node [black,anchor=west,xshift=10pt] at (#1) {#2};
    }
}

\newcommand{\blockcomment}[3]{%
    \tikz[remember picture, overlay]{
        \draw [decorate,decoration={lineto,amplitude=10pt,mirror,raise=4pt},yshift=0pt,very thick,{#3}] 
        (#1) -- (#2) node [black,midway,xshift=10pt] {};
    }
}



\usepackage{biblatex}
\addbibresource{references.bib}

\begin{document}         
\author{Lorenzo Tucci, Reyhaneh Rabaninejad}
\title{Secure cross-chain decentralized exchange}

\maketitle

\tableofcontents
\newpage

\begin{todobox}
\section*{Current Status and meeting notes}

\noindent \textbf{{Threshold RingCT}} The approach is to thresholdize the tag in RingCT. I am now reading Russell's thesis to understand the technical details on this.
\begin{itemize}
    \item Construct tag from a \textbf{distributed PRF}.
    \item Generate aggregated proof for proving that the partial PRFs and partial key pairs are consistent.
   \end{itemize}
\noindent \textbf{{Pool setting}}
    The most trivial setting is one in which all liquidity providers setup the same amount on all the chains supported in the system. If we want to support users participating only in a subset of the pools, additionally completexity is created because of the need to track balances for each liquidity provider (or key share) participating in the swaps. 
    More literature review might be needed to compare to existing DEX/AMM protocols.

\noindent \textbf{Remarks.}
\begin{enumerate}
    \item {We consider private AMM and mempool privacy out of scope.}
    \item Any number of ledgers can be involved, as long as all parties can use a standard smart contract (e.g., Ethereum) on a ledger $L_{Ex}$.
    \item In this protocol, there is no timelock assumption, in contrast to atomic swap protocols, which additionally require a timelock script.
\end{enumerate}

\end{todobox}

% TeX root = main.tex

\section{Introduction}

\subsection{Our Contributions}

\subsection{Related Work}

% TeX root = atomic-swaps.tex

\section{Preliminaries}

\subsection{Basic primitives}

\begin{todobox}
    commitments, ZKP, ...
\end{todobox}
\subsection{Non-Interactive Zero Knowledge Proofs}

Let $R: \{0, 1\}^* \times \{0, 1\}^* \rightarrow \{0, 1\}$ be a NP-witness-relation with corresponding NP-language $\mathcal{L} := \{x : \exists w \:\: \text{s.t.} \:\: R(x, w) = 1\}$

A non-interactive zero-knowledge proof (NIZK) system for R consist of the following algorithms:
\begin{itemize}
    \item $\mathsf{cr} \gets \mathsf{ZK}_\mathcal{L}.\mathsf{Setup}(1^\lambda)$ takes on input the security parameter, outputs a common reference string $\mathsf{crs}$
    \item $\pi \gets \mathsf{ZK}_\mathcal{L}.\mathsf{Pr}(\mathsf{crs}, x, w)$ takes on input the reference string $\mathsf{crs}$, a statement $x$ and a witness $w$, outputs a proof $\pi$
    \item $0/1 \gets \mathsf{ZK}_\mathcal{L}.\mathsf{Vr}(\mathsf{crs}, x,\pi)$ takes on input the reference string $\mathsf{crs}$, a statement $x$ and a proof $\pi$. Outputs 1 if $w$ is a witness for the statment $x$, 0 otherwise.
\end{itemize}
We require a NIZK system to be \textit{zero-knowledge}, where the verifier does not learn more than the validity of the statement $x$, and \textit{simulation sound} where it is hard for any prover
to convince a verifier of an invalid statement (chosen by the prover) even after having access to polynomially many simulated proofs for statements of his choosing.

\subsection{Secure 2-Party Computation}
A secure 2-party computa-tion (2PC) protocol allows two participating users $P_0$ and $P_1$ to securely compute some function $f$ over their private inputs $x_0$ and $x_1$ respectively.

We require the standard $privacy$ property, which states that the only information learned by the parties in the computation is that specified by the function output.
We also require the standard security with aborts, where the adversary can decide whether the honest party will receive the output of the computation or not, and thus there are no assumptions on fairness or guaranteed output delivery.

\subsection{Computational Assumptions}

\begin{todobox}
    General setting of group-based crypto, implicit notation, assumptions that we need e.g. DLOG 
\end{todobox}
\newpage

% TeX root = atomic-swaps.tex

\section{Decentralized exchange}

\subsection{Syntax}
\subsection{Correctness}
\subsection{Security}
\subsection{Construction}

\newpage

% TeX root = atomic-swaps.tex

\section{Threshold RingCT transactions}

\subsection{Syntax}
\begin{definition}
A threshold ring confidential transactions (\textsf{tRingCT}) scheme consists of the PPT algorithms
\[
\Omega^{(t,n)} = (\textsf{Setup}, \textsf{tKGen}, \textsf{tKDer}, \textsf{pTrans}, \textsf{Vf}, \textsf{StExt}, \textsf{TxExt}, \textsf{SrcChk}, \textsf{TgtChk})
\]
with the following interfaces:

\begin{itemize}
    \item $\mathsf{Setup}(1^\lambda) \rightarrow (\mathsf{pp}, \mathsf{st})$: Initializes the system by generating public parameters $\mathsf{pp}$ and an initial state $\mathsf{st}$, using the security parameter $\lambda$.

    \item $\mathsf{tKGen}(\mathsf{pp}, n, t) \rightarrow \{(\mathsf{mpk}_j, \mathsf{msk}_j)\}_{j=1}^n$: Produces $n$ public/private key pairs distributed across $n$ participants, such that any subset of at least $t$ can jointly derive usable secrets. The result includes key shares $(\mathsf{mpk}_j, \mathsf{msk}_j)$ for each party.

    \item $\mathsf{KDer}(\mathsf{msk}_j, \mathsf{tk}) \rightarrow (\mathsf{sk}_j, a)$: On input private key share $\mathsf{msk}_j, {i \in J}$ and a token $\mathsf{tk}$, party $P_j$ derives a secret key $\mathsf{sk}_j$ and corresponding amount $a$.

    \item $\mathsf{pTrans}(\mathsf{st}, P, R, \mathcal{S}, \mathcal{T}) \rightarrow (\mathsf{tx}_j, \mathsf{TK})$: Constructs a transaction $\mathsf{tx}_j$ along with a set of target tokens $\mathsf{TK} = \{\mathsf{tk}_i\}_{i \in \mathcal{T}}$. It takes as input the current state $\mathsf{st}$, a policy predicate $P$ over amounts, a ring of anonymity set indices $R$, \textcolor{red}{a set of} source accounts information $\mathcal{S} = \{(\mathsf{sk}_i, a_i)\}$, and a target specification $\mathcal{T} = \{(\mathsf{mpk}'_i, a'_i)\}$.

    \item $\mathsf{Combine}(\{\mathsf{tx}_j\}_{j \in J}, \mathsf{TK}) \rightarrow (\mathsf{tx}, \mathsf{TK})$:  This algorithm aggregates a set of partial transactions $\{\mathsf{tx}_j\}_{j \in J}$—generated independently by authorized parties—into a single valid transaction $\mathsf{tx}$. The input also includes the set of target tokens $\mathsf{TK}$, which are preserved or merged as needed in the final output. The resulting transaction $\mathsf{tx}$ must satisfy structural integrity and verification constraints as if it had been produced by a single-party execution of $\mathsf{Trans}$.


    \item $\mathsf{Vf}(\mathsf{st}, \mathsf{tx}) \rightarrow (b, \mathsf{st}')$: Verifies the validity of the transaction $\mathsf{tx}$ with respect to the current state $\mathsf{st}$. If valid, returns $b = 1$ and an updated state $\mathsf{st}'$; otherwise returns $b = 0$.

    \item $\mathsf{StExt}(\mathsf{st}) \rightarrow \mathsf{AC}_U$: Extracts the complete set of accounts $\mathsf{AC}_U = \{\mathsf{ac}_i\}$ currently recorded in the state.

    \item $\mathsf{TxExt}(\mathsf{tx}) \rightarrow \mathsf{AC}_T$: Extracts from a transaction $\mathsf{tx}$ the list of target accounts $\mathsf{AC}_T = \{\mathsf{ac}_i\}$ that were newly introduced or referenced.

    \item $\mathsf{SrcChk}(\mathsf{ac}, \mathsf{sk}_j, a) \rightarrow b$: \textcolor{red}{Verifies that a source account} $\mathsf{ac}$ is associated with the given secret key $\mathsf{sk}_j$ and amount $a$. Outputs $1$ if valid, else $0$.

    \item $\mathsf{TgtChk}(\mathsf{ac}, \mathsf{mpk}, \mathsf{tk}, a) \rightarrow b$: Checks whether a target account $\mathsf{ac}$ was properly generated under public key $\mathsf{mpk}$ and token $\mathsf{tk}$ for the value $a$. Outputs $1$ if the check passes, and $0$ otherwise.
\end{itemize}
\end{definition}

\subsection{Correctness}
\subsection{Security}
\subsection{Construction}
In general, the design involves two types of transactions: threshold-based and standard (non-threshold) transactions. Within the context of a DEX, freeze transactions temporarily lock user funds by transferring them to addresses governed by threshold signature schemes. This mechanism ensures that no single party can unilaterally access the funds. Instead, a quorum of threshold servers must collaboratively generate valid signatures to authorize subsequent transactions, thereby finalizing the exchange securely and collectively. 

\newpage

% TeX root = atomic-swaps.tex

\section{Instantiation and Performance Evaluation}

\subsection{Instantiation}

\paragraph{Commitment}

\paragraph{Tagging Scheme}

\paragraph{2PC}

\paragraph{Zero-Knowledge Proofs}

\subsection{Performance Evaluation}

\end{document}



\printbibliography

\end{document}

