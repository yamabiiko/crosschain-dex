% TeX root = main.tex

\section{Introduction}

An atomic cross-chain swap is a cryptographic protocol that enables two mutually distrustful parties to exchange assets residing on separate blockchains, ensuring the \textit{atomicity} property: either both parties receive the corresponding asset from the other, or no transfer takes place. This guarantee must hold even in the presence of network delays or adversarial behavior, ensuring that neither party can cause a loss of funds for the other. To achieve this, atomic swap protocols rely on mechanisms that place each participant’s asset into a conditional transfer state—locked in such a way that it can be claimed by the counterparty upon satisfying specific conditions (e.g., revealing a secret or solving a puzzle), or refunded to the sender after a designated timeout. \\
The predominant approach for implementing atomic swaps is via Hash Time-Locked Contracts (HTLCs), which combine hash preimages with time-locks to enforce atomicity. While effective, HTLCs require on-chain scripting capabilities, a requirement that is incompatible with many privacy-focused cryptocurrencies such as Monero or ZCash. These systems intentionally restrict or eliminate scripting functionality to enhance privacy and efficiency, rendering HTLCs infeasible in their context. \\
This limitation poses a significant barrier to interoperability: despite the growing demand for cross-chain trading, privacy-preserving blockchains remain largely isolated due to the lack of compatible atomic swap mechanisms. Prior works, such as Universal Atomic Swaps (UAS), have explored protocols that replace hash-locks with time-lock puzzles (TLPs) to eliminate scripting assumptions. While theorically sound, TLP-based protocols face serious practical challenges. In particular, their security critically depends on potential difference in (linear) computational power, which may be of several orders of magnitude faster with certain hardware capabilities (e.g., ASICs). As a result, tuning puzzle difficulty to balance security and usability leads to protocols that are either impractical for honest users or insecure against adversaries with specialized hardware.
This motivates the development of an atomic swap protocol that requires minimal on-chain functionality, making it suitable for privacy-focused blockchains like Monero and ZCash, which lack scripting capabilities, yet still supporting efficient and secure cross-chain asset transfers.

\subsection{Our Contributions}
In this work, we propose a new minimal blockchain functionality called \emph{commit transactions} which enables supporting blockchains to perform atomic swap through minimal and efficient protocols. Our main contributions are outlined below.
\paragraph*{Commit transactions.} We propose \emph{commit transactions}, a general-purpose blockchain functionality for expressing time-based conditional asset transfers. Commit accounts allow users to lock funds under dynamic ownership conditions, such as transitioning from shared control to unilateral recovery based on a predefined timeout. The functionality is compatible with both UTXO and account-based ledger models and can be integrated into public ledgers with minimal modification to the transaction verification logic, requiring no general-purpose scripting and incurring no additional transaction size overhead.
\paragraph*{RingCCT (Ring Confidential Transaction).} We extend the RingCT model to the commit transaction functionality. The scheme incorporates commit accounts with shared ownership semantics and time-conditional spending rules, while preserving the confidentiality and anonymity guarantees of standard RingCT. We present a concrete instantiation of RingCCT based on succinct non-interactive arguments, commitment schemes, and key-homomorphic pseudorandom functions.
\paragraph*{Practical atomic swaps.} 
Leveraging the commit transaction functionality, we design an efficient and secure generic protocol for secure cross-chain swaps. We further show how to construct practical atomic swap protocols that can be executed between any pair of ledgers—whether both are public blockchains, both are RingCCT-based, or a mix of the two.


\subsection{Related Work}
