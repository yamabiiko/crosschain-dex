\section*{Current Status and meeting notes}

\noindent \textbf{{Threshold RingCT}} The approach is to thresholdize the tag in RingCT. I am now reading Russell's thesis to understand the technical details on this.
\begin{itemize}
    \item Construct tag from a \textbf{distributed PRF}.
    \item Generate aggregated proof for proving that the partial PRFs and partial key pairs are consistent.
   \end{itemize}
\noindent \textbf{{Pool setting}}
    The most trivial setting is one in which all liquidity providers setup the same amount on all the chains supported in the system. If we want to support users participating only in a subset of the pools, additionally completexity is created because of the need to track balances for each liquidity provider (or key share) participating in the swaps. 
    More literature review might be needed to compare to existing DEX/AMM protocols.

\noindent \textbf{Remarks.}
\begin{enumerate}
    \item {We consider private AMM and mempool privacy out of scope.}
    \item Any number of ledgers can be involved, as long as all parties can use a standard smart contract (e.g., Ethereum) on a ledger $L_{Ex}$.
    \item In this protocol, there is no timelock assumption, in contrast to atomic swap protocols, which additionally require a timelock script.
\end{enumerate}
